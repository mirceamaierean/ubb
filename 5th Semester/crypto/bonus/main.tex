\documentclass[12pt]{article}

% Packages
\usepackage{amsmath, amsthm, amssymb, amsfonts}
\usepackage{mathtools}
\usepackage{geometry}                            % To adjust margins
\usepackage{enumitem}                            % For custom lists
\usepackage{hyperref}                            % For hyperlinks
\geometry{a4paper, margin=0.8in}                   % Set page margins
\DeclarePairedDelimiter\floor{\lfloor}{\rfloor}

% Theorem, Lemma, Corollary, and Proof Environments
\newtheorem{theorem}{Theorem}[section]
\newtheorem{lemma}[theorem]{Lemma}
\newtheorem{corollary}[theorem]{Corollary}

% Definition and Remark
\theoremstyle{definition}
\newtheorem{definition}[theorem]{Definition}
\newtheorem{example}[theorem]{Example}
\newtheorem{remark}[theorem]{Remark}
\newtheorem{proposition}[theorem]{Proposition}

% Title, Author, and Date
\title{Continued Fraction Factorisation Method}
\author{Mircea Măierean}
\date{\today}

% Document Begin
\begin{document}

\maketitle

% Abstract
\begin{abstract}
This paper presents the required steps for finding a factor of a number using the Continued Fraction Method (CFRAC). Initially, the theoretical aspects of the method are presented, followed by an exemplification on a natural number.
\end{abstract}

% Table of Contents (optional)
\tableofcontents


% Introduction
\section{Definitions}
In this section, we will define some of the terms that will be used across this paper, as well as their corresponding notations
\begin{definition}
A continued fraction is of the form
\begin{equation}%
\large % (b)
x = \frac{q_0}{q_1 + \frac{1}{q_2 + \frac{1}{q_3 \dots}}}%
\end{equation}
\textit{We can also identify continued fractions in sequence form as 
\[ x = [q_0; q_1, q_2, \ldots], q_i \in \mathbb{Z}, i \in \mathbb{N} \]
}
The above is as an example of an \textit{infinite continued fraction}; a \textit{finite continued fraction in sequence form is \[ x = [q_0; q_1, q_2, \ldots,q_n], n \in \mathbb{N}, q_i \in \mathbb{Z}, i = \overline{1, n}
\]
}
\end{definition}

\begin{definition}
    The $k$-th convergent of a finite continued fraction is $[q_0; q_1, q_2 \ldots q_k], k\in \mathbb{N}$
\end{definition}

\begin{definition}
    A \textit{periodic infinite continued fraction} corresponds to an irrational number $x$, where $x = [q_0; q_1, q_2 \ldots q_j, \overline{q_{j + 1}, q_{j + 2}, \ldots q_{j + p}}$. Here, $p$ denotes the periodicity of the terms repeated. 
\end{definition}

\begin{definition}
    Let $N$ be a positive integer that is not a square. The $n$-th complete quotient of $x_n$, where $x_n$ is the $n$-th convergent of $\sqrt{N}$ is defined as
\begin{equation}
\large
x_n = \begin{cases} 
\sqrt{N}, &\text{if } n = 0, \\\
\frac{1}{x_{i-1} - q_{i-1}}, &\text{if } n \geq 1.
\end{cases}
\end{equation}
With respect to $x_n$, $q_n=\floor{x_n}$
\end{definition}

\begin{definition}
Let $P$ and $Q$ be $2$ sequences defined by the following recurrences: 

    \begin{equation}
        \large
        P_n = \begin{cases} 
            0, & \text{if } n = 0, \\\
            q_0, & \text{if } n = 1, \\\
            q_{n - 1} Q_{n - 1} - P_{n - 1}, & \text{if } n \geq 2.
        \end{cases}
    \end{equation}
    
    \begin{equation}
        \large
        Q_n = \begin{cases} 
            1, & \text{if } n = 0, \\\
            N - q_0^2, & \text{if } n = 1, \\\
            Q_{n - 2} + (P_{n - 1} - P_{n})q_{n - 1}, & \text{if } n \geq 2.
        \end{cases}
    \end{equation}
\end{definition}

\begin{definition}
    Let $(-1)^nQ_n=Q^*_n$. Two $Q^*_n$'s are equivalent if their product is a square, that is, $Q^*_i$ is equivalent to $Q^*_j$ if $x^2Q^*_i=y^2Q_j^*$, for $x, y \in \mathbb{Z}$
\end{definition}

% Preliminaries or Notation (optional)
\section{Theorems}
In this section, the theorems that are used will be listed below, with their proper citations.
\begin{theorem}
For a positive non-square integer $N$, the period starts after the first term in the continued fraction for $\sqrt{N}$, i.e $\sqrt{N}=[q_0;\overline{q_1, q_2,\ldots, q_{p - 1}, 2q_0}]$. Moreover, the sequence $q_1, q_2, \ldots, q_{p - 1}$ has the property that $q_{p-i}=q_i, i = \overline{1, p - 1}$ \cite{rose}
\end{theorem}

% Main Theorems and Proofs
\section{Propositions}

\begin{proposition}
    For any positive integer $N$ that is not a square, the statement:
    \begin{equation}
        A(n): N = P_n^2+Q_nQ_{n-1}, \forall n \in \mathbb{N^*}
    \end{equation}
    is always true
\end{proposition}

\begin{proof}

This statement will be proven using mathematical induction.

\textbf{Base case:} Let \(n = 1\). \\
We check that the statement holds for \(n = 1\):
\[
A(1): N = P_1^2 + Q_1 Q_0 = q_0^2 + N - q_0^2 = N.
\]
Thus, the base case holds.

\vspace{1em}

\textbf{Inductive Step:} \\
Assume that the statement holds for some arbitrary \(k\), i.e.,
\[
A(k): N = P_k^2 + Q_k Q_{k - 1}.
\]
We need to show that \(A(k + 1)\) also holds:
\[
A(k + 1): N = P_{k + 1}^2 + Q_{k + 1} Q_k.
\]
Based on the statements, we have:
\[
P_k^2 + Q_k Q_{k - 1} = P_{k + 1}^2 + Q_{k + 1} Q_k \Longleftrightarrow
\]
\[
P_k^2 - P_{k + 1}^2 = Q_{k + 1} Q_k - Q_k Q_{k - 1} \Longleftrightarrow
\]
\[
(P_k - P_{k + 1})(P_k + P_{k + 1}) = Q_k (Q_{k + 1} - Q_{k - 1}) \Longleftrightarrow
\]
\[
(P_k - P_{k + 1}) q_k Q_k = Q_k (Q_{k + 1} - Q_{k - 1}) \Longleftrightarrow
\]
\[
(P_k - P_{k + 1}) q_k = Q_{k + 1} - Q_{k - 1} \Longleftrightarrow
\]
\[
(P_k - P_{k + 1}) q_k = Q_{k - 1} + (P_k - P_{k + 1}) q_k - Q_{k - 1} \Longleftrightarrow
\]
\[
(P_k - P_{k + 1}) q_k = (P_k - P_{k + 1}) q_k.
\]

Since this holds for any \(k \in \mathbb{N}\), the inductive step is proven.

\vspace{1em}

By the principle of mathematical induction, the statement \(A(n)\) holds for all \(n \geq 1\).
\end{proof}

\begin{proposition}
    For any positive integer $N$ that is not a square, the statement:
    \begin{equation}
        B(n): x_n = \frac{\sqrt{N} + P_n}{Q_n}, \forall n \in \mathbb{N}
    \end{equation}
    is always true
\end{proposition}

\begin{proof}

This statement will be proven using mathematical induction.

\textbf{Base case:} Let \(n = 0\). \\
We check that the statement holds for \(n = 0\):
\[
B(0): x_0 = \frac{\sqrt{N} + P_0}{Q_0} = \frac{\sqrt{N} + 0}{1}=\sqrt{N}
\]
Thus, the base case holds.

\vspace{1em}

\textbf{Inductive Step:} \\
Assume that the statement holds for some arbitrary \(k\), i.e.,
\[
B(k): x_k = \frac{\sqrt{N} + P_k}{Q_k}.
\]
We need to show that \(B(k + 1)\) also holds:
\[
B(k + 1): x_{k + 1} = \frac{\sqrt{N} + P_{k + 1}}{Q_{k + 1}}.
\]
From the definition of $x$, we have:
\[
x_{k + 1} = \frac{1}{x_k-q_k} \Longleftrightarrow
\]
\[
x_{k + 1} = \frac{1}{\frac{\sqrt{N} + P_k}{Q_k}-q_k} \Longleftrightarrow
\]
\[
x_{k + 1} = \frac{1}{\frac{\sqrt{N} + P_k - q_k Q_k}{Q_k}} \Longleftrightarrow
\]
\[
x_{k + 1} = \frac{Q_k}{\sqrt{N} - (q_k Q_k - P_k)} \Longleftrightarrow
\]
\[
x_{k + 1} = \frac{Q_k}{\sqrt{N} - P_{k+1}} \Longleftrightarrow
\]
\[
x_{k + 1} = \frac{Q_k(\sqrt{N} + P_{k+1})}{N-P_{k+1}^2}
\]

Based on \textbf{Proposition 3.1}, $N-P_{k+1}^2 = Q_k * Q_{k + 1}$, thus resulting in
\[
x_{k + 1} = \frac{Q_k(\sqrt{N} + P_{k+1})}{Q_k * Q_{k + 1}} \Longleftrightarrow
\]
\[
x_{k + 1} = \frac{\sqrt{N} + P_{k+1}}{Q_{k + 1}}
\]


Since this holds for any \(k \in \mathbb{N}\), the inductive step is proven.

\vspace{1em}

By the principle of mathematical induction, the statement \(B(n)\) holds for all \(n \geq 0\).
\end{proof}

\begin{proposition}
\textbf{The P Method}

    For any positive integer $N$ that is not a square, the statement:
    \begin{equation}
        C(n): (-1)^nQ_n(P_{n-1}P_{n-3}P_{n-5}\ldots P_r)^2 \equiv (P_nP_{n-2}P_{n-4} \ldots P_s) ^2 \pmod N,
    \end{equation}
    where $r = 1$ and $s = 2$ if $n$ is even and reversed otherwise, is always true, $\forall n \in \mathbb{N^*}$
\end{proposition}

\begin{proof}

This statement will be proven using mathematical induction. 

From \textbf{(5)}, we have
\[-Q_nQ_{n-1} \equiv P_n^2 \pmod N\]

\textbf{Base case:} Let \(n = 1\). \\
We check that the statement holds for \(n = 1\):
\[
C(1): -Q_1 \equiv P_1^2 \pmod N \Longleftrightarrow -Q_1Q_0 \equiv P_1^2 \pmod N \Longleftrightarrow q_0^2-N \equiv q_0^2 \pmod N 
\]
Thus, the base case holds.

\vspace{1em}

\textbf{Inductive Step:} \\
Assume that the statement holds for some arbitrary \(k\), i.e.,
\[
C(k): (-1)^kQ_k(P_{k-1}P_{k-3}P_{k-5}\ldots P_r)^2 \equiv (P_kP_{k-2}P_{k-4} \ldots P_s) ^2 \pmod N,
\]
We need to show that \(C(k + 1)\) also holds:
\[
C(k + 1): (-1)^{k+1}Q_{k + 1}(P_kP_{k - 2}P_{k - 4}\ldots P_s)^2 \equiv (P_{k + 1}P_{k-1}P_{k - 3} \ldots P_r) \pmod N,
\]
\[
(-1)^kQ_k(P_{k-1}P_{k-3}P_{k-5}\ldots P_r)^2 \equiv (P_kP_{k-2}P_{k-4} \ldots P_s) \pmod N \Longleftrightarrow
\]
\[
(-1)^{k+1}Q_{k + 1}Q_k(P_{k - 1}P_{k - 3}P_{k - 5}\ldots P_r)^2 \equiv -Q_{k + 1}(P_kP_{k - 2}P_{k - 4} \ldots P_s) ^ 2 \pmod N \Longleftrightarrow
\]
\[
-Q_{k + 1}(P_kP_{k - 2}P_{k - 4} \ldots P_s) ^ 2 \equiv (-1)^{k}(-Q_{k + 1}Q_k)(P_{k - 1}P_{k - 3}P_{k - 5}\ldots P_r)^2 \pmod N \Longleftrightarrow
\]
\[
-Q_{k + 1}(P_kP_{k - 2}P_{k - 4} \ldots P_s) ^ 2 \equiv (-1)^k(P_{k + 1}P_{k - 1}P_{k - 3}P_{k - 5}\ldots P_r)^2  \pmod N \Longleftrightarrow
\]
\[
(-1)^{k + 1}Q_{k + 1}(P_kP_{k - 2}P_{k - 4} \ldots P_s) ^ 2 \equiv (P_{k + 1}P_{k - 1}P_{k - 3}P_{k - 5}\ldots P_r)^2  \pmod N 
\]

Since this holds for any \(k \in \mathbb{N^*}\), the inductive step is proven.

\vspace{1em}

By the principle of mathematical induction, the statement \(C(n)\) holds for all \(n \geq 1\).
\end{proof}
% Example Section
\section{Finding the factors}
The assigned number is $N=7861$. For this number, we will compute the corresponding values for $q_n, P_n$ and $Q_n^*$ $\pmod N$, using equations $(2)$, $(3), (4)$. Initially, we need to compute $q_0 = \floor{\sqrt{n}} = \floor{\sqrt{7861}} = 88$. The values obtained are:
\begin{table}[h!]
    \centering
    \begin{tabular}{|c|c|c|c|}
        \hline
         \textbf{$n$} & \textbf{$P_n$} & \textbf{$Q_n^*$} & \textbf{$q_n$} \\
        \hline
         0  &  0  &  1  &  88  \\
         1  & 88  & -117 &  1   \\
         2  & 29  &  60 &  1   \\
         3  & 31  & -115 &  1   \\
         4  & 84  &   7 &  24  \\
         5  & 84  &- 115 &  1   \\
         6  & 31  &  60 &  1   \\
         7  & 29  & -117 &  1   \\
         8  & 88  &   1 & 176  \\
         9  & 88  & -117 &  1   \\
        10  & 29  &  60 &  1   \\
        11  & 31  & -115 &  1   \\
        12  & 84  &   7 &  24  \\
        13  & 84  & -115 &  1   \\
        14  & 31  &  60 &  1   \\
        15  & 29  & -117 &  1   \\
        16  & 88  &   1 & 176  \\
        17  & 88  & -117 &  1   \\
        18  & 29  &  60 &  1   \\
        19  & 31  & -115 &  1   \\
        20  & 84  &   7 &  24  \\
        21  & 84  & -115 &  1   \\
        22  & 31  &  60 &  1   \\
        23  & 29  & -117 &  1   \\
        24  & 88  &   1 & 176  \\
        25  & 88  & -117 &  1   \\
        26  & 29  &  60 &  1   \\
        27  & 31  & -115 &  1   \\
        \hline
    \end{tabular}
    \caption{Table of values for $P_n$, $Q_n^*$, $q_n$, and corresponding results.}
    \label{tab:values}
\end{table}

Since $7861$ is a positive integer that is not a square, based on \textbf{(2.1)}, we can represent it as a \textit{\textbf{periodic infinite continued fraction}}:
\[
7861 = [88; \overline{1, 1, 1, 24, 1, 1, 1, 176}]
\]
Moreover, we observe that, after the $8th$ iteration, $176$ appears, which is exactly $2 * 88$. We can stop iterating after this value appears, as everything will be repeated, but for the purpose of visualizing data we have iterated a few more steps.

The goal now is to obtain 2 squares that have the same congruence mod $N$, which can lead to a possible solution. \textbf{(7)} provides in its corresponding equation already a square on both sides as a factor. Unfortunately, none of the values for $Q^*$ are squares, so we will try to find 2 equations, $i, j$ such that $Q_i^* Q_j^*$ will also form a square. 

Looking at Table 1, $Q_{3}^*$ and $Q_{5}^*$, besides having the same parity for their indexes, they have the same sign, so we obtain:
\[
    Q_3^*P_2^2 \equiv (P_3P_1)^2 \pmod N
\]
and
\[
    Q_{5}^*(P_{4}P_{2})^2 \equiv (P_{5}P_{3}P_{1})^2 \pmod N
\]

Multiplying these 2 equations, a new congruence is obtained:
\[
    (115P_{4}P_2^2)^2 \equiv (P_5P_3^2P_1^2)^2 \pmod N
\]

Let $t_1 = 115P_{4}P_2^2$ and $t_2 = P_5P_3^2P_1^2$.

\[
t_1= 8124060 \longrightarrow t_1^2=66000350883600 \longrightarrow t_1^2 \equiv 7658 \pmod{7861}
\]

\[
t_2= 625126656 \longrightarrow t_2^2 =390783336041742336 \longrightarrow t_2^2\equiv 7658 \pmod{7861}
\]

The previous relation can be written as well as $(t_1 + t_2) * (t_2 - t_1) \equiv0 \pmod N$. One of the $\gcd(t_1 + t_2, N)$ and $\gcd(t_2 - t_1, N)$ might be a proper factor of $N$. We will compute them accordingly.

\[\gcd (t_1 + t_2, N) = \gcd(8124060 + 625126656, 7861) = \gcd(633250716, 7861) = 7861
\]
which is not a proper factor, since it is equal to $N$.

\[\gcd (t_2 - t_1, N) = \gcd(617002596, 7861) = 7
\]
which corresponds to a proper factor.

Moreover, we can find the other factor as well, $\frac{7861}{7} = 1123$, which is also a prime number.

Concluding, $7861 = 7 * 1123$

If we would have obtained both values of the $\gcd$'s invalid, then we would have needed to retry the whole process, by selecting different equations for  $Q_i^*$ and $Q_j^*$.

% Conclusion
\section{Conclusion}
In this paper, we have applied the theoretical aspects of \textbf{The P Method} for finding the factors of 7861. We have found that $7 * 1123 = 7861$.

% References
\begin{thebibliography}{99}

\bibitem{rose} H. E. Rose, \emph{A Course in Number Theory}, Oxford Science Publications, 2nd Edition, pg. 130, 1994.


\end{thebibliography}

\end{document}
